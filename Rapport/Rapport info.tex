\documentclass[11pt,a4paper]{article}

\usepackage{latexsym}
\usepackage{graphicx}
\usepackage[french]{babel}


\usepackage{amsmath,amssymb}
\usepackage{pstricks,pst-plot}
\usepackage{calc}
\usepackage{multicol}
\usepackage{fancyhdr}
\usepackage{lastpage}
\usepackage[T1]{fontenc}
\usepackage[utf8]{inputenc}  
\usepackage{lmodern}
\usepackage{stmaryrd}
\usepackage[]{algorithm2e}
\usepackage{float}

\pagestyle{plain}

\title{Projet d'Informatique: SudokuSolver}
\author{Mathurin \textsc{Massias} \and Clément \textsc{Nicolle}}
\date{\today} 


\begin{document}
	
\maketitle

Nous préférons vous écrire ce court rapport afin de vous spécifier les quelques libertés que nous avons prises par rapport au sujet original.
\\
\begin{itemize}
	\item Nous avons pris les conventions anglaises (W au lieu de O par exemple).
	\\
	\item Nous avons vérifié les premières méthodes dans le \textit{main.cpp} de \textit{TestSudoku}, puis avons finalement utilisé des tests unitaires. Vous les trouverez dans le projet \textit{SudokuUnitTest}, fichier \textit{unittest1.cpp}. Nous obtenons une code coverage de 95,23 \%.
	\\
	\item Pour accéder aux cellules et aux régions, nous avons implémenté des accesseurs basés sur les coordonées.
	\\
	\item Avec les seules stratégies proposées, nous ne pouvions pas résoudre la grille de la Partie 3. Nous avons donc ajouté un visiteur \textit{RowColumnRegion} qui vérifie pour chaque cellule s'il n'y aurait pas qu'une seule valeur possible compte-tenu de sa région, ligne et colonne.
	\\
	\item La dernière grille, "diabolique", n'est pas résoluble. Nous l'avons vérifié sur un autre solver. Afin de tester la stratégie basée sur les hypothèses implémentée dans la Partie 4, nous avons utilisé une grille customisée partagée au sein de l'option (nous avons vérifié qu'on ne peut la résoudre juste avec les stratégies intermédiaires).
\end{itemize}
%
\hspace{3mm}
\\
Pour runner les tests, il faut d'abord builder \textit{SudokuSolver} (projet par défaut), puis builder la solution.

\end{document}